\documentclass{article}

\usepackage{amsmath,amsthm,amssymb}
\usepackage[T1,T2A]{fontenc}
\usepackage[utf8]{inputenc}
\usepackage[english,russian]{babel}
\usepackage[all]{xy}
\usepackage[left=2cm,right=2cm, top=2cm,bottom=2cm,bindingoffset=0cm]{geometry}
\usepackage{mathtext}

\renewcommand{\Im}{\operatorname{Im}}
\newcommand{\Perm}{\operatorname{S}}
\newcommand{\sgn}{\operatorname{sgn}}
\newcommand{\Mat}{\operatorname{M}}
\usepackage[12pt]{extsizes} % для того чтобы задать нестандартный 14-ый размер шрифта


\title{Аннотация к научно исследовательской работе}
\author{Подоляко Илья Александрович БПМИ-176}
\date{}
\begin{document}
\maketitle

\begin{center}
Иерархическое планирование поведения
\end{center}

\paragraph{}

Темой исследования в настоящей работе является изучение алгоритмов иерархического планирования. \newline
Цель работы - рассмотреть основные алгоритмы планирования, в частности алгоритм HEART, и сравнить его эффективность работы с алгоритмами реализованными в планнировщике pyperplan. \newline
Основным методом проведения исследования является создание программы на языке Python. Также будут использоваться научные материалы по проблематике планирования поведения искусствнного агента. Алгоритм планнирования HEART основан на иерархической версии алгоритма POCL и алгоритма HTN. Процесс планирования алгоритма POCL осуществлятеся с использованием частично уточнённых планов состоящих из шагов и каузальных ссылок на уже найденные подпланы. HTN использует различные уровни иерархии действий представляя планы низкого уровня в виде абстрактных действий более высокого уровня.
 Также будет использован планнировщик pyperplan, в котором реализованы известные подходы к планнированию поведения, что позволит сравнить алгоритм HEART уже с существующими алгоритмами.\newline
Результатом работы должна быть разработанная программа реализующая логику алгоритма HEART. \newline
Предполагается, что в итоге будет достигнута корректная работа алгоритма в реальных условиях. 

\newpage
\paragraph{Список источников}
\paragraph{Стюарт Рассел, Питер Норвиг}
Искусственный интеллект. Современный подход / 2-е изд.. Ж Пер. с англ. - М. : Издательский дом "Вильямс", 2006. -  1408 с. : ил. - Парал. тит. англ. ISBN 5-8459-0887-6 (рус.) \newline
\paragraph{Antoine Gréa, Laetitia Matignon, and Samir Aknine}
 - HEART: HiErarchical Abstraction for Real-Time Partial Order Casual Link Planning / Proceedings of the $1^{st}$ ICAPS Workshop on Hierarchical Planning
\paragraph{Malik Ghallab, Dana Nau, and Paolo Traverso} 2004 Automated Planning and Acting, Cambridge University Press? editions. 00058
\paragraph{Subbarao Kambhampati, Amol Mali, and Biplav Srivastava.} 1998/ Hybrid planning for partially hierarchial domains. In AAAI/IAAI, pages 882-888
\paragraph{Dana S. Nau, Tsz-Chui Au, Okhtay Ilghami, Ugur Kuter, J. William Murdock, Dan Wu, and Fusun Yaman.} 2003. SHOP2: An HTN planning system. J. Artif. Intell. Res(JAIR), 20:379-404. 00891
\paragraph{}ГОСТ Р 7.0.5–2008. Система стандартов по информации, библиотечному и
издательскому делу. Библиографическая ссылка Общие требования и
правила составления. М.– Стандартинформ, 2008.
\paragraph{}ГОСТ 7.1-2003. Библиографическая запись. Библиографическое описание.
Общие требования и правила составления.- М.: Изд-во стандартов, 2003.
\paragraph{}ГОСТ 7.82-2001. Библиографическая запись. Библиографическое описание
электронных ресурсов.- М.: Изд-во стандартов, 2001.

\end{document}