\documentclass{article}

\usepackage{amsmath,amsthm,amssymb}
\usepackage[T1,T2A]{fontenc}
\usepackage[utf8]{inputenc}
\usepackage[english,russian]{babel}
\usepackage[all]{xy}
\usepackage[left=2cm,right=2cm, top=2cm,bottom=2cm,bindingoffset=0cm]{geometry}
\usepackage{mathtext}
\usepackage{xcolor}
\usepackage{hyperref}

\renewcommand{\Im}{\operatorname{Im}}
\newcommand{\Perm}{\operatorname{S}}
\newcommand{\sgn}{\operatorname{sgn}}
\newcommand{\Mat}{\operatorname{M}}

\title{Иерархическое планирование поведения}
\author{Подоляко Илья Александрович БПМИ-176}
\date{}
\begin{document}

\maketitle



\section*{Версии PDDL и их различия.}

В данной части рассмотрим некоторые отличия PDDL разных версий. Стоит сказать, что главными особенностями была поддержка языка для STRIPS-планировщика(не языка). Т.е. формально это попытка объединить все хорошие особенности предыдущих языков(как я понял, ADL и STRIPS-язык) и создать единый формат, который бы был хотя бы совместим со STRIPS-планировщиком.(в связи с тем, что STRIPS-язык не поддерживает условыне эффекты были бы проблемы, но PDDL имеет декларацию о условных эффектах и мы сразу понимаем о его совместимости). Также многие планировщики имеют дополнительную аннотацию называемую советами(advice - подсказка куда что использовать и др), но так как это загромождает описание и обязательно не для каждого планировщика, то создатели PDDL решили свести это к минимуму и поэтому создана отдельная синтаксическая конструкция. Девиз PDDL - physics, not advice, что означает, что приоритет отдан физике(Домену), описывающей предикаты действий, состояний и тд. \newline 
Сразу оговорим особенности, чтобы не писать их в таблице первой версии(PDDL 1.2), а описать в таблице непосредственно изменения следующих версий. \newline
Во-первыx, PDDL описывает домены планирования(физика наших задач). Домен декларирует типы объектов в нашей системе, виды отношений между объектами и их свойства, константы системы, схемы действий и схемы аксиом. \newline
Во-вторых, PDDL декларирует задачу планирования(task). Это разделение удобно тем, что для одной и той же системы всегда можно применить разные таски, без введения данных о мире заново. В задаче декларируют принадлежность к домену, множество объектов системы, начальное состояние системы и конечное состояние. \newline
По мимо этого PDDL ещё разрешает декларировать следующие конструкции: \newline
1) Флюент; (Флюента (fluent) — это функция, областью определения которой является множество всех возможных ситуаций.) \newline
2) Вычисляемых выражений; \newline
3) Условных эффектов действий; \newline
4) Иерархических структур действий (для иерархических планировщиков); \newline
5) Ограничений сохранности (Своего рода дополнительных целей, которые должны быть истинны в целевом состоянии, если они были истины в начальном состоянии. Если ограничение сохранности нарушается где-то в процессе планирования, то оно должно быть восстановлено в дальнейшем.); \newline
6) Ограничений на длину целевого плана.
\begin{center}
\begin{tabular}{|c|c|c|c|}
\hline
PDDL 2.1 & PDDL 2.2 & PDDL 3.0\\
\hline
1) Поддержка старых версий&1) Выводимые литералы(derived predicates)&1) Появились средства для\\
(При условии корректности там)&(по сути аксиомы, т.е. если некий p& описания дополнительных\\
2) Работа метрик для оценки& имеет key и key подходит к safe,y& ограничений (допустимость\\
 качества плана& то p может открыть safe.)& или присутствие тех\\
3) Доменные функции - &2) Отложеные начальные литералы& или иных действие в плане)\\
позволяют работать с числовыми& (timed initial literals) - & 2)Описание предпочтений -\\
единицами(переливание воды)&средство для декларации внешних& желательные цели и ограничения.\\
4) Параллелизм действий - & событий, которые начнутся& Они не обязательны в исполнении\\
самолёт летит, и теряет&по истечении некоторого времени,& но с помощью них появляется\\
топливо одновременно& известного планировщику. & возможность качественно оценить\\
5) Убраны аксиомы&& результат.\\
\hline
\end{tabular}
\end{center}
Наиболее точное описание синтаксиса и особенностей лежит здесь(честно взято) 
\href{http://ai-center.botik.ru/planning/index.php?ptl=materials/03languages.htm}{тык.}

\section*{Другие способы описания.}
Первым стоит вспомнить обычный STRIPS язык, который не совсем является языком. Формально это просто вошло в следующие года как обозначение формальных языков, но у него так же имеется некоторое определение и описание задач планирования. \newline
Второй это ADL, который является фактически предком PDDL. Особенность является в том, что по сути был внедрён принцип открытости  мира, в отличии от того же STRIPS'а. Также появляется поддержка отрицательных предикатов, незивестные предикаты имеют неизвестное состояние, а не отрицательное и тд. Всё в итоге ведёт к тому, что это переосмысление проблем STRIPS.

\section*{Сравнение алгоритмов поиска и их описание}
Описание алгоритмов библиотеки pyperplan: \newline
1) А* это фактически Дейкстра,улучшенная эвристической функцией. Эвристика состоит в том, что штраф за проход к вершине высчитывается не как расстояние до вершины, а расстояние до вершины + расстояние от данной врешины до конечной вершины(на самом деле дополнительный штраф может быть и дргуим, всё зависит от выбранной нами функции $h(x)$). \newline
2) WASTAR - это A* только с заданным весом. \newline
3) BFS - алгоритм поиска в ширину, переходим в соседнии вершины по очереди и записываем их в очередь. Делаем так, пока не придём в нашу целевую вершину. \newline
4) Enforced hill climbing search - жадный алгоритм выбирающий наиболее перспективную вершину по соседству. \newline
5) Iterative deeping search - итеративный алгоритм DFS, немного эффективнее, чем обычный DFS. Он также опитмален как и DFS, но различие в том, что выбор следующей вершины происходит как и в BFS. \newline
6) Greedy best first search - алгоритм поиска первого лучшего. Ищем наиболее лучшего соседа в нашей эвристике, и идём просто к нему. \newline
Будем смотреть на следующеие факты: время работы, количество узлов, длинна плана. (Далее Wall-clock search time, Nodes expanded, Plan length соответственно). Наиболее приоритетный длинна плана, после время работы, после кол-во узлов.

\begin{center}
\begin{tabular}{|c|c|}
\hline
Алгоритм поиска&Результаты на blocks\\
\hline
A*&23 Nodes expanded\\
&Wall-clock search time: 0.061\\
&Plan length: 12\\
\hline
WASTAR&26 Nodes expanded\\
&Wall-clock search time: 0.038\\
&Plan length: 12\\
\hline
BFS&3353 Nodes expanded\\
&Wall-clock search time: 0.16\\
&Plan length: 12\\
\hline
Enforced hillclimbing search&24 Nodes expanded\\
&Wall-clock search time: 0.035\\
&Plan length: 16\\
\hline
Iterative deeping search&14306 Nodes expanded\\
&  Wall-clock search time: 0.56\\
&Plan length: 12\\
\hline
Greedy best first search&25 Nodes expanded\\
&Wall-clock search time: 0.03\\
&Plan length: 12\\
\hline
\end{tabular}
\end{center}  

Вывод: \newline
Наиболее GBFS, что довольно необычно, учитывая как работает алгоритм. EFS составил самый длинный план и притом с наибольшей затратой времени, что на самом деле нормально, учитывая, что это жадный алгоритм. Дольше всех работал IDS, при том больше всего вершин создано.

\section*{Сравнение эвристическиих алгоритмов и их описание.}

В данной части рассмотрим эвристики в библиотеке pyperplan и сравним их на эффективность. Для большей точности я рассматривал две задачи: первая airport(задача планирования вылета самолётов из аэропортов) и вторая sokoban(игра сокобан). Задачи выглядят максимально разными в выборке. Параметры для эвристик: \newline.
Будем смотреть на следующеие факты: время работы, количество узлов, длинна плана. (Далее Wall-clock search time, Nodes expanded, Plan length соответственно). Наиболее приоритетный длинна плана, после время работы, после кол-во узлов. Используется стандартный алгоритм поиска в данной библиотеке hff.

\begin{center}
\begin{tabular}{|c|c|c|c|}
\hline
Эвристика & Описание & Резльтат на airport & Результат на sokoban\\
\hline
Blind&Суть простая, если& 11 Nodes expanded &2551 Nodes expanded\\
(вслепую)&достигли цели - 1,& Wall-clock search time: 0.00058 &Wall-clock search time: 0.15\\
&иначе 0.&  Plan length: 8 &Plan length: 49\\
\hline
Landmarks&Данная эвристика основана&9 Nodes expanded&2941 Nodes expanded\\
(ориентиры)&на разделении ситуаций&Wall-clock search time: 0.00072&Wall-clock search time: 0.2\\
&без которых не достигнуть&Plan length: 8&Plan length: 49\\
&цели. Эти ориентиры,&&\\
&они имеют больший вес.&&\\
\hline
LM-cut&Идея схожа с &9 Nodes expanded&95 Nodes expanded\\
(разрез ориентир.)&прошлой, но различие в&Wall-clock search time: 0.011&Wall-clock search time: 1.1\\
&том, что теперь ориентиры& Plan length: 8&Plan length: 49\\
&отрезают от начала.&&\\
\hline
Relaxation&Данный метод содержит&9 Nodes expanded&103 Nodes expanded\\
(Релаксация)&в себе несколько доп.алго.&Wall-clock search time: 0.0032&Wall-clock search time: 0.14\\
&Я выбрал лучший результат&Plan length: 8&Plan length: 49\\
&из каждой категории,&&\\
&ибо все релаксации&&\\
&основаны на том, что&&\\
&все кратчайшые подпути - &&\\
&есть участки кратчайшего пути.&&\\
\hline
\end{tabular}
\end{center}

\paragraph*{Выводы:}
Были выбраны две задачи, с разными размерами. Размер плана во всех ситуациях получился одинаковым(возможно тогда оптимальным). На маленьких тестах лучшим получился обычный алгортим вслепую, но при этом он заэкспандил больше всех узлов. В больших же тестах ситуация получилась немного другая. Лидером по времени оказалась релаксация, меньше всего заэкспанжено узлов у LM-cut. Что действительно удивительно в данной ситуации, что Blind дал такой быстрый результат, при там кол-ве узлов. 

\end{document}
